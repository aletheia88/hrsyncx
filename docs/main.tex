\documentclass[11pt]{article}
\usepackage[margin=1in]{geometry}
\input{configs}
\usepackage{setspace}

\title{Mechanisms of neuronal synchronization in Hindmarsh-Rose network with time-varying connectivity}

\begin{document}
\maketitle

\begin{abstract}
This project investigates how time-varying long-range connectivity shapes
neuronal synchronization in a network of Hindmarsh--Rose (HR) model neurons
coupled via electrical synapses on a small-world Watts--Strogatz graph. Using a
fourth-order Runge-Kutta integrator implemented in JAX, we simulate networks of
$N=50$ or $N=200$ HR neurons for up to $5\times 10^5$ time steps while
systematically varying the rewiring probability $p_r$, the power-law decay
exponent $\alpha$ governing long-range interactions, and the coupling strength
$\varepsilon$. Synchrony is quantified by a time-averaged Euclidean distance
between neuronal state vectors. After identifying and correcting a subtle
implementation error in the Watts--Strogatz rewiring algorithm, our simulations
reproduce the key qualitative findings of Rakshit et al.: increasing $p_r$
robustly promotes synchrony, larger $\alpha$ (stronger suppression of
long-range couplings) hinders or delays global synchronization, and stronger
$\varepsilon$ facilitates near-complete synchrony with errors as low as $E
\approx 0.03$. These results highlight how dynamic small-world topology and
long-range interactions jointly control emergent synchronization in HR networks
and provide a reusable JAX-based framework for exploring synchronization
mechanisms in more realistic brain-inspired circuits.
\end{abstract}

\section{Introduction}
% Synchronization as a subject field
Synchrony in neuronal networks plays a dual role as a crucial mechanism of
healthy brain functions underlying cognition and motor coordination, but also a
potential indicator of pathological states, such as schizophrenia, Parkinson's
disease and epilepsy \cite{Majhi2025Patterns}. Importantly, synchronous
neuronal dynamics is an emergent phenomenon arising from the interacting
dynamics of multiple neuronal and non-neuronal cells, including both pairwise
interactions and higher-order correlations in neuronal ensembles
\cite{Majhi2025Patterns}. Past literature suggest various forms of neuronal
synchrony can be driven by the connectivity mechanisms. The algebraic
connectivity of a network (i.e., the second minimal eigenvalue of Laplace
matrix) affects the synchronization of nonlinear networks under diffusive
coupling \cite{Plotnikov2021Synchronization}. Through spike-timing-dependent
plasticity (STDP), neurons receiving coherent input could temporally associate
\cite{Gansel2022Neural}. This project focuses on the synchronization in
networks of Hindmarsh-Rose systems. 

% HR as a model of synchronization
The Hindmarsh-Rose (HR) system describes a biological neuron in terms of a nonlinear
system of third-order differential equation with three variables
\cite{Hindmarsh1984Model}:
\begin{align}
  \dot{x} &= -a x^3 + b x^2 + y - z + I \label{eq:org.x} \\
  \dot{y} &= c - dx^2 - y \label{eq:org.y} \\ 
  \dot{z} &= r(s(x+w)-z) \label{eq:org.z}
\end{align}
where $x$ represents the dynamics of membrane potential, while $y$ describes
the dynamics of fast sodium currents and $z$ describes the dynamics of slow
potassium currents. The Hindmarsh-Rose (HR) model based on pairwise neuronal
interactions has been used to understand brain network synchronization in
children with ADHD, which showed that ADHD brains react more rapidly and
strongly to emotional stimuli \cite{Ansarinasab2023Synchronization}.

My aim here is to explore the synchronizing impact of long-range connections
under time-varying connectivity via electrical synapses in the HR model. The
primary goal is to replicate/confirm the core finding from the Ref.
\cite{Rakshit2021Neuronal} of the dependence of synchronization on coupling
strength and long-range interaction delays in order to enhance my own
understanding of the subject.

\section{Method}

\subsection{Hindmarsh-Rose Model}

Suppose a network of $N$ neurons connected through temporal long-range
electrical synapses. Let $x_i$ represent the membrane potential of the $i$-th
neuron, $y_i$ represents to the fast $Na^+$ or $K^+$ current, and $z$ represent
the slow $Ca^+$ current. The network according to the Ref.
\cite{Rakshit2021Neuronal} can be mathematically expressed as the following:
\begin{align}
  \dot{x_i} &= y_i - a x_i^3 + b x_i^2 - z_i + I + \epsilon
    \sum_{k=1}^{k_\text{max}(t)} \frac{1}{k^{\alpha}} \sum_{j=1}^N
  \mathcal{A}_{ij}^{[k]}(t) (x_j-x_i) \label{eq:main.x}\\
  \dot{y_i} &= c - d x_i^2 - y_i \label{eq:main.y} \\
  \dot{z_i} &= r(s(x_i - x_0) - z_i) \label{eq:main.z}, \ i=1,2,\dots,N.
\end{align}
In the equation, $\epsilon$ is the coupling strength, $r$ is the modulator of
the slow dynamics. The model parameters are chosen as $a=1$, $b=3$, $c=1$,
$d=5$, $r=0.005$, $I=3.25$, $s=4$ and $x_0 = -1.6$ \cite{Rakshit2021Neuronal}.
In addition, $\mathcal{A}_{ij}^{[k]}$ represents the adjacency matrix at time
$t$, characterizing temporal connectivity in the network, where $k$ describes
the range of interactions. If the $i$th and $j$th nodes are $k$-path connected,
the element $\mathcal{A}_{ij}^{[k]}(t) = 1$ and $\mathcal{A}_{ij}^{[k]}(t) = 0$
otherwise. At each time instant $t$, the entire network is rewired with
probability $p_r$. Hence, the adjacency matrix $\mathcal{A}_{ij}^{[k]}(t)$ for
each $k$-th path network is time-dependent. The $i$th and $j$th nodes are
considered $k$-path connected if $\mathcal{A}_{ij}^{[k]}(t)=1$ and $1 \leq k
\leq k_{max}(t)$, where $k_{max}(t)$ is the diameter of the network at the time
instance $t$, or the largest shortest-path length between any pair of nodes at
that time. In other words, the entries of the time-varying $k$-path adjacency
matrix $\mathcal{A}_{ij}^{[k]}(t)$ for $1 \leq k \leq k_{max}(t)$ are as
follows:
\begin{equation}
  \mathcal{A}_{ij}^{[k]}(t) = 
  \begin{cases}
    1, & \text{if distance$(i,j)=k$ at time $t$} \\
    0, & \text{otherwise}
  \end{cases}
\end{equation}
The effective coupling strength $\epsilon_k = \epsilon / k^{\alpha}$ follows a
power-law decay, where $\alpha$ controls the decay rate. The underlying network
connectivity is set to observe a small-world architecture accroding to the
Watts-Strogatz (WS) graph model. The entire network, with the architecture of a
ring lattice, is rewired with probability $p_r$. Small $p_r$ implies that the
network is almost static, while a large value of $p_r$ indicates a rapidly
varying network. Initially, each node in the network is adjacent with $k_{sw}$
nearest neighbors on both sides, so the degree of each node is $2k_{sw}$. These
initial links are then reconnected randomly to chosen distant nodes with
probability $p_{sw}$. That means, for each edge on the ring, with probability
$p_{sw}$, we delete that edge and reconnect the source node to a random node,
avoiding self-loop and duplicating edges. 

\subsection{Synchronizaton Error}

The error of synchronization will be averaged over a sufficiently large time
duration ($10^5$ time steps based on the Ref. \cite{Rakshit2021Neuronal} as
follows:
\begin{equation}
  E = \left\langle \frac{1}{N-1} \sum_{j=2}^N \sqrt{(x_j-x_i)^2
  + (y_j-y_i)^2+(z_j-z_i)^2} \right\rangle_t
\end{equation}
where $N$ is originally chosen to be 200 HR neurons. The evolution of $E$ is
examined by varying network and coupling parameters to observe the conditions
under which synchronization is promoted. In the original paper, phase diagrams
were created to illustrate the dependence of synchronization on the coupling
strength $\epsilon$ and the small-world parameter $p_{sw}$ for varying levels
of $\alpha$ in Eq.~\ref{eq:main.x}, which control the long-range interaction
decay. 

\section{Experiments}
% Describe simulation procedures
The original paper \cite{Rakshit2021Neuronal} simulated 200 HR neurons for $3
\times 10^5$ time steps. Due to computing resource limit, we either reduced the
network size to $N=50$ or the number of time steps to $1 \times 10^5$.
Additionally, we simulated the network in JAX for speed optimization. Equation
\ref{eq:main.x} \ref{eq:main.y} \ref{eq:main.z} were manually integrated using
the fourth order Runge-Kutta method over $3 \times 10^5$ time iterations, with
$dt = 0.01$ as the integration time step. Specifically, the dynamics of $y$ and
$z$ were simulated as written in Eq.~\ref{eq:main.y} and Eq.~\ref{eq:main.z}.
Let $F(x,y,z) = y_i - a x_i^3 + b x_i^2 - z_i + I$. And we call
\[
  B_{ij}(t) = \sum_{k=1}^{k_{max}(t)} \frac{1}{k^{\alpha}} A_{ij}^{[k]}(t)
\]
the effective coupling term. Then we can rewrite Eq.~\ref{eq:main.x} as follows:
\begin{align}
  \dot{x}_i &= F(x, y, z) + \epsilon \sum_{k=1}^{k_{max}(t)} \frac{1}{k^{\alpha}} \sum_j A_{ij}^{[k]}(t)(x_j - x_i) \\
  \dot{x}_i &= F(x, y, z) + \epsilon \sum_j B_{ij} (x_j - x_i) \\
  \dot{x}_i &= F(x, y, z) + \epsilon \left[ \sum_j B_{ij}x_j - x_i \sum_j B_{ij} \right] \label{eq:sim.x}
\end{align}
We use Eq.~\ref{eq:sim.x} to simulate for $x$. For initial conditions, we tried
randomly drawing from the normal distribution using \texttt{jax.random.normal};
we also tried sampling from the states of the original HR model (defined by
Eq.~\ref{eq:org.x}, Eq.~\ref{eq:org.y} and Eq.~\ref{eq:org.z}) after the first
$50,000$ time steps, where the trajectory is in a regime exhibiting chaotic
attractor dynamics.

Following the Ref. \cite{Rakshit2021Neuronal}, we investigated the
synchronization error subject to the variation of the parameters $p_{sw}$,
$p_r$ and $\epsilon$. Interestingly, our simulation resulted very different
error dynamics, probably due to differences in the length of simulation and
initial conditions.

\subsubsection{Experiment I: Effect of $\mathbf{\alpha}$ on synchronization}

\paragraph{Fix $\mathbf{p_{sw}=0.1, k_1 = 3, \epsilon=0.5}$, varying $\mathbf{\alpha, p_r}$.}
Expect to see increasing $p_r$ promotes neuronal synchronization irrespective
of the chosen values of $\alpha$, though $\alpha$ still significantly affects
where synchronization emerges. Specifically, the higher the value of $\alpha$,
the higher the value of $p_r$ required to achieve complete synchronization
where $E \approx 0$.

\paragraph{Primary Results} We first simulated the network with $N=200$, experimenting
with different state initializations, either drawing $(x_0, y_0, z_0)$ from
chaos (Fig.~\ref{fig1:chaos}) or from normal distribution
(Fig.~\ref{fig1:normal}). Our simulation in Fig~\ref{fig1:pr_vs_E} shows as
$p_r$ increases, synchronization error decreases and stablizes under both
initializations, approaching perfect synchrony, which is consistent with our
anticipation. However, unlike what the Ref.\cite{Rakshit2021Neuronal} claims in
their paper, the $\alpha$ value does not seem to strongly modulate where
synchronization occurs. 

\begin{figure}[htbp]
    \centering
    % --- left image ---
    \begin{subfigure}{0.48\textwidth}
        \centering
        \includegraphics[width=\linewidth]{figures/wrong_wiring/fig1-200N_init-from-chaos.jpg}
        \caption{Initialize from chaos.} % (a)
        \label{fig1:chaos}
    \end{subfigure}
    \hfill
    % --- right image ---
    \begin{subfigure}{0.48\textwidth}
        \centering
        \includegraphics[width=\linewidth]{figures/wrong_wiring/fig1-200N.jpg}
        \caption{Initialize from normal distribution.} % (b)
        \label{fig1:normal}
    \end{subfigure}

    \caption{Synchronization error $E$ as a function of $p_r$ for different $\alpha$.}
    \label{fig1:pr_vs_E}
\end{figure}

\paragraph{Additional Observations} When we reduced the network size to $N=50$
but increased the stimulation time steps to $5 \times 10^5$, $E$ no longer
exhibits a decreasing trend. Still we did not see $\alpha$ differentiates the
synchronization dynamics under the normal distribution initialization.

\begin{figure}[htbp]
    \centering
    % --- left image ---
    \begin{subfigure}{0.48\textwidth}
        \centering
        \includegraphics[width=\linewidth]{figures/wrong_wiring/fig1-50N.jpg}
        \caption{Network simulation with $N=50$, $T=5000$, $dt=0.01$.} % (a)
        \label{fig1.1:50N}
    \end{subfigure}
    \hfill
    % --- right image ---
    \begin{subfigure}{0.48\textwidth}
        \centering
        \includegraphics[width=\linewidth]{figures/wrong_wiring/fig1-200N-100T.jpg}
        \caption{Network simulation with $N=200$, $T=100$, $dt=0.01$.} % (b)
        \label{fig1.1:100T}
    \end{subfigure}

    \caption{Synchronization error $E$ as a function of $p_r$ for different $\alpha$.}
    \label{fig1.1:pr_vs_E}
\end{figure}

\subsubsection{Experiment II: Effect of $\mathbf{\epsilon}$ on synchronization}

\paragraph{Fix $\mathbf{\alpha=2.5, k_{sw}=3, p_{sw}=0.1}$, varying $\mathbf{\epsilon, p_r}$.}
Expect to see an increasing $p_r$ favors synchrony. In constrast to the effect
of $\alpha$ on synchronization, here we expect to observe increasing values of
$\epsilon$ enhances the appearance of neuronal synchrony.

\paragraph{Primary Results} We simulated an HR network of size $N=200$ for $1
\times 10^5$ time steps starting from an initial state randomly sampled from
trajectory in a chaotic regime according to the classic HR model. The effect of
$p_r$ is consistent with our expectation: higher $p_r$ values lead to lower
$E$, because as the network wiring switches more rapidly, the neurons get the
opportunity of interacting with the otherwise-noninteracting neurons
\cite{Rakshit2021Neuronal}. But we still do not see an obvious pattern in how
the network modulation parameters affect synchronization, unlike what the Ref.
\cite{Rakshit2021Neuronal} predicted. At this point, we think it is reasonable
to suspect that our simulation may have some critical difference with the
original HR network described. We will delve into this in the Discussion section.

\begin{figure}[htbp]
  \centering
  \includegraphics[width=0.7\linewidth]{figures/wrong_wiring/fig2-200N_init-from-chaos.jpg}
  \caption{Synchronization error $E$ as a function of $p_r$ for different $\epsilon$}
  \label{fig2:pr_vs_E}
\end{figure}

\section{Discussion}

After scrutinizing over our code, we found one minor but critical mistake in
our implementation of the canonical Watts-Strogatz (WS) graph model, which the
original paper \cite{Rakshit2021Neuronal} unfortunately does not describe in
sufficient detail. In the canonical WS model for an undirected graph, the
adjacency (or connectivity) matrix starts from a regular ring lattice, where
each node $i$ is connected to its $k$ nearest neighbors. For each edge $(i, j)$
in some ordered subset, with probability $p_{sw}$, it gets rewired according to
the following rules: (1) remove edge $(i, j)$; (2) choose a new node $j'$
uniformly at random from nodes not equal to node $i$ and not already connected
to $i$; (3) add edge $(i, j')$. Under this algorithm, the number of edges and
the degree of every node are \emph{always preserved} (plus, there are no
self-loops or multi-edges). In our implementation, when updating the
connectivity of an existing edge with probability $p_{sw}$, we failed to put
any constraint on where the new edges could be drawn from in addition to
inhibiting self-loops. Suppose there are $N$ nodes in the network. As a result,
the new node $j'$ can be any node $0, \dots, N-1$. If $j'=i$, we would skip
adding any new edge (i.e., the original edge is removed and \textbf{not
replaced}), which leads to decreasing the total edge count in the graph, thus
changing the degree of the node. If $j'$ is already a neighor of $i$, then we
would remove edge $(i, j)$ and add a "new" edge that already exists, which
effectively results in loosing one edge and decreasing the degree of both $i$
and $j$. Therefore, the degree distribution is slightly changing over time. To
fix this, we instead chose the new node $j'$ uniformly from the set $\{ 0,
\dots, N-1\} \ / \ (\{i\} \cup \text{neighbors}(i))$.

We re-ran experiment I to explore the effect of $\alpha$ on synchronization.
Previously with the erroneous implementation of the WS graph model, we did not
observe $\alpha$ modulating when network synchronization occurs, and the
network never achieved perfect synchrony (i.e., $E\approx0$) with any of the
$(p_r, \alpha)$ combinations we tried. Initializing a network with $N=200$ and
simulated for $1 \times 10^5$ iterations yielded an oscillating $E \approx 2.3$
for all $\alpha's$ as we increased $p_r$. We first simulated a network with
$N=50$ for $5 \times 10^5$ iterations, initialized from
\texttt{jax.random.normal} (Fig.~\ref{fig3:pr_vs_E}). We found exactly what the
Ref. \cite{Rakshit2021Neuronal} concluded: higher $\alpha$ resulted in higher
synchronization error; in addition, the network with only 50 HR neurons were
able to achieve $E\approx 0.08$, a significant reduction from the previous
$E\approx2.3$. 

\begin{figure}[htbp]
    \centering
    % --- left image ---
    \begin{subfigure}{0.48\textwidth}
        \centering
        \includegraphics[width=\linewidth]{figures/fig1-50N_noskip.jpg}
        \caption{Network re-simulation with $N=50$, $T=5000$, $dt=0.01$.} % (a)
        \label{fig3:50N_noskip}
    \end{subfigure}
    \hfill
    % --- right image ---
    \begin{subfigure}{0.48\textwidth}
        \centering
        \includegraphics[width=\linewidth]{figures/fig1-50N_skip.jpg}
        \caption{Zooming in on Fig.~\ref{fig3:50N_noskip}, skipping the first $p_r$.} % (b)
        \label{fig3:50N_skip}
    \end{subfigure}

    \caption{Synchronization error $E$ as a function of $p_r$ for different $\alpha$.}
    \label{fig3:pr_vs_E}
\end{figure}

Next, we simulated a larger network with $N=200$ for $1 \times 10^5$
iterations, initialized from chaos (Fig.~\ref{fig4:pr_vs_E}). With more
neurons, the network has synchronization error further reduced to
$E\approx0.03$, achieving almost perfect synchrony, while we still observed
that increasing $\alpha$ hinders synchrony, just as the Ref.
\cite{Rakshit2021Neuronal} had shown. 

\begin{figure}[htbp]
    \centering
    % --- left image ---
    \begin{subfigure}{0.48\textwidth}
        \centering
        \includegraphics[width=\linewidth]{figures/fig1-200N_init-from-chaos_noskip.jpg}
        \caption{Network re-simulation with $N=200$, $T=1000$, $dt=0.01$.} % (a)
        \label{fig4:200N_noskip}
    \end{subfigure}
    \hfill
    % --- right image ---
    \begin{subfigure}{0.48\textwidth}
        \centering
        \includegraphics[width=\linewidth]{figures/fig1-200N_init-from-chaos_skip.jpg}
        \caption{Zooming in on Fig.~\ref{fig4:200N_noskip}, skipping the first $p_r$.} % (b)
        \label{fig4:200N_skip}
    \end{subfigure}

    \caption{Synchronization error $E$ as a function of $p_r$ for different $\alpha$.}
    \label{fig4:pr_vs_E}
\end{figure}

Experiment II explores the effect of $\epsilon$ on synchronization. The Ref
\cite{Rakshit2021Neuronal} observed that increasing $\epsilon$ enhances
synchrony, opposite to the effect of increasing $\alpha$. We simulated a
network of $N=200$ for $1 \times 10^5$ time steps, initialized from chaos and
integrated with the fourth order Runge-Kutta as before. Our results beautifully
agree with this conclusion (Fig.~\ref{fig5:pr_vs_E}). 

\begin{figure}[htbp]
  \centering
  \includegraphics[width=0.6\linewidth]{figures/fig2-200N_init-from-chaos_skip.jpg}
  \caption{Synchronization error $E$ as a function of $p_r$ for different $\epsilon$}
  \label{fig5:pr_vs_E}
\end{figure}


\section{Appendix}

All code, results and figures can be found in
\url{https://github.com/aletheia88/hrsyncx/tree/main}.

\clearpage
{
\small
\bibliographystyle{apalike}
\bibliography{references}
}

\end{document}
